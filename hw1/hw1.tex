\documentclass[11pt,a4paper]{article}

\usepackage{amsfonts}
\usepackage{amssymb}

\usepackage{amsthm}
\usepackage{epsfig}
\usepackage{graphicx}
\usepackage{natbib}		% citet, citep
\usepackage{textcomp}
\usepackage{booktabs}
\usepackage{multirow}
\usepackage{fullpage}
\usepackage{authblk}
\usepackage{url}
\usepackage{color}
\usepackage{tikz}
% \usepackage{xeCJK}
% \setCJKmainfont{TW-Kai} 

% use for game tree
\usepackage{amsmath}
\def\vpay#1#2{\begin{matrix}#1\\#2\end{matrix}}
\usepackage{istgame}

\renewcommand{\baselinestretch}{1.4}

\parskip=5pt
\parindent=20pt
\footnotesep=5mm

\newtheorem{lem}{Lemma}
\newtheorem{prop}{Proposition}
\newtheorem{defn}{Definition}
\newtheorem{cor}{Corollary}
\newtheorem{ass}{Assumption}
\newtheorem{obs}{Observation}
\newenvironment{pf}{\begin{proof}\vspace{-10pt}}{\end{proof}}
% \newtheorem{ques}{Question}
% \newtheorem{rmk}{Remark}
% \newtheorem{note}{Note}
% \newtheorem{eg}{Example}

\newenvironment{enumerateTight}{\begin{enumerate}\vspace{-8pt}}{\end{enumerate}\vspace{-8pt}}
\newenvironment{itemizeTight}{\begin{itemize}\vspace{-8pt}}{\end{itemize}\vspace{-8pt}}
\leftmargini=25pt   % default: 25pt
\leftmarginii=12pt  % default: 22pt

\DeclareMathOperator*{\argmax}{argmax}
\DeclareMathOperator*{\argmin}{argmin}
\setcounter{MaxMatrixCols}{20}

\title{Computer Network and Application (110-1) \\ Homework 1}

\author{Michael Chen (B08705051)}
\date{}

\begin{document}

\maketitle

\section{R4}

\subsection{Home Access}
\begin{itemize}
    \item Digital subscriber line (DSL)
    \item Fiber to the home 
    \item Cable internet access
    \item Wi-Fi
\end{itemize}

\subsection{Enterprise Access}
\begin{itemize}
    \item Ethernet
\end{itemize}

\subsection{Wide-area Wireless Access}
\begin{itemize}
    \item 3G and 4G
\end{itemize}

\section{R13}

\subsection{a.}
2 users.

\subsection{b.}
Because if there's two or fewer user transmit at the same time, the link capacity won't be fully occupied, so there won't be any queuing delay. \\
But if three or more users are transferring simultaneously, the transmitted data would exceed the link's max capacity, so there will be queuing delay.

\subsection{c.}
20 percent.

\subsection{d.}
The probability is $0.2^3 = 0.008$ \\
The probability of the queue growing is $0.2 + 0.2^2 + 0.2^3 + .... + 0.2^\infty = \frac{1}{4}$

\section{R16}
\begin{itemize}
    \item Packet transmission delay : a constant delay time $\frac{L}{R}$ 
    \item Queuing delay : a variable delay determined by how many packets are waiting for the output link
    \item Propagation delay : constant delay time
    \item Processing delay : constant delay time
\end{itemize}

\section{P3}

\subsection{a.}
A circuit switched network would be more appropriate for this application, since the transmission rate is steady and has a long session, we can reserve 
bandwidth for the application without waste and save the time of constantly setting up new connections.

\subsection{b.}
No congestion control is needed. Because even if all applications transmit data at the same time, it still won't exceed the max capacity of the link.

\section{P6}

\subsection{a.}
$d_{\text{prop}} = m / s $

\subsection{b.}
$d_{\text{trans}} = L / R$

\subsection{c.}
The end to end delay is $d_{\text{prop}} + d_{\text{trans}} =  m / s + L / R$.

\subsection{d.}
The bit would be just leaving Host A at time $t = d_{\text{trans}}$

\subsection{e.}
Because $d_{\text{prop}} > d_{\text{trans}}$, so the first bit would be in the link, hasn't reached Host B yet.

\subsection{f.}
The bit would have reached Host B.

\subsection{g.}
$\frac{m}{2.5 \cdot 10^8} = \frac{120}{56 \cdot 10^3}$ \\
$m = 2.5 \cdot 10^8 \cdot \frac{120}{56 \cdot 10^3} = \frac{75}{14} \cdot 10^5 = \frac{7500}{14}$ km

\section{P7}
How long it needs to gather one packet : $\frac{56 \cdot 8}{64 \cdot 10^3} = 0.007$ sec \\
Transfer time : $\frac{56 \cdot 8}{2 \cdot 10^6} = 0.000224$ sec \\
Total time : $0.007 + 0.000224 + 0.01 = 0.017224$ sec 

\section{P8}

\subsection{a.}
$\frac{3 \cdot 10^6}{150 \cdot 10^3} = 20$ users

\subsection{b.}
0.1

\subsection{c.}
$C^{120}_n \cdot 0.1^n \cdot 0.9^{120-n} $

\subsection{d.}
Ans = 1 - Probability of less than or equal to 20 users are transmitting data simultaneously
$1 - \sum^{20}_{i=0} C^{120}_i  \cdot 0.1^i \cdot 0.9^{120-i}$

\section{P31}

\subsection{a.}
$\frac{8 \cdot 10^6}{2 \cdot 10^6} = 4$ sec \\
There are 3 links, so $4 \times 3 = 12$ seconds

\subsection{b.}
$\frac{10000}{2 \cdot 10^6} = 0.005$ sec \\
Suppose the first packet is sent at t=0 (in seconds), the second packet will reach the first switch at t=0.01.

\subsection{c.} 
1. \\
After the first packet reached the destination, the rest 799 packets will arrive one after another with a 0.005 interval. \\
So the total time is $0.005 \cdot 3 + 799 \cdot 0.005 = 4.01$ sec. \\
2.\\
The message segmentation method has significantly less delay.

\subsection{d.}
Without message segmentation, if there's a packet loss, the whole message has to be transferred again. \\
Second, these large packets may exceed the router's storage space, and thus will never reach its destination.

\subsection{e.}
With message segmentation, the message would be split in many small segments all with its own header, which would lead to 
more header bytes.
 

\end{document}